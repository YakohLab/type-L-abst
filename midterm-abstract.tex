\documentclass[twocolumn,9pt]{ltjsarticle}
\usepackage{graphicx}
\usepackage{amsmath,amssymb}
\usepackage{mathtools}
	\mathtoolsset{showonlyrefs=true}
\usepackage{./midterm-abstract}
% Lualatexの処理系の問題で、siunitxを使う場合は\usepackage{./midterm-abstract}の後にusepackageする
\usepackage[detect-all]{siunitx}

\def\thesisnumber{2018-XX-YY}
\title{和文題目}
\def\englishtitle{English Title}
\author{8ZZZZZZZ 慶應太郎 (Taro Keio) Supervisor: 矢上諭吉 (Yukichi Yagami)}
\begin{document}
\maketitle
\section{緒論}
レイアウトと文字サイズ等について示す.
左右余白は20mm,上下余白は20mmとする.

題目は,1段組で中央配置とし,ゴシック,太字,14ptとする.
下に適当な間隔を開け英文題目を記載する.
さらにその下に適当な間隔を開け,学籍番号(半角),氏名(欧文),Supervisor 教員氏名(欧文)をゴシック,太字,12ptで1段組,中央配置で記載する(欧文字もゴシック).

その下に適当な間隔を開け,本文を記載する.
本文は,2段組,明朝,9pt,1行27字程度,1コラム55~60行程度を標準とする.
章は,9pt,ゴシック,太字,左寄り印刷とする.
節と項は,9pt,ゴシック,太字,左寄り印刷とする.
英数字は,Timesを用いることを基本とする.
字体に意味がある場合(ベクトルなど)には,この限りではない.
\end{document}
