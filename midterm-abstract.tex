\documentclass[twocolumn,9pt]{ltjsarticle}

\usepackage{graphicx}
\usepackage{amsmath,amssymb}
\usepackage{mathtools}
    \mathtoolsset{showonlyrefs=true}
\usepackage{./midterm-abstract}
% Lualatexの処理系の問題で、siunitxを使う場合は\usepackage{./midterm-abstract}の後にusepackageする
\usepackage[detect-all]{siunitx}
\usepackage{tabularx}
\usepackage{mwe}
\usepackage{cleveref}
\renewcommand{\ref}{\Cref}
\addto\captionsenglish{\renewcommand{\figurename}{Figure~}}
\addto\captionsenglish{\renewcommand{\tablename}{TABLE~}}

\def\thesisnumber{2019-16-02}
\title{マルチユーザ赤外線通信の設計}
\def\englishtitle{Designing a Multi-user Infrared Communication}
\author{81816518 鳥野見武家 (Takeie Torinomi) Supervisor: 矢向高弘 (Takahiro Yakoh)}
\begin{document}
\maketitle
\section{Introduction}
One common lightwave telecommunication method is infrared, widely used in remote controllers for consumer electronics, and previously used as an intre-device communication.
Although, since infrared communication runs at a low modulation frequency of \SI{38}{kHz}, obtaining a compromisable data rate is a challenge. 
Ever since the improvement of radio frequency (RF) telecommunication, it has not drawn much attention.

Nonetheless, we argue that however, it was impossible with any existing protocols, by carefully designing a new protocol, we can achieve a reasonable data rate, with multiuser characteristics.
Multiuser characteristics are essential when considering an open environment, where many terminals enter and leave the communication channel on the fly.

In this paper, we propose a tightly packed frame protocol, enabling multiuser access in an open environment.
Our system is implemented as a system on a chip (SoC), enabling us an appropriate balance of performance and flexibility.

\section{Conventional protocols}
To today, many proposed their protocol for their intended use, each with advantages and disadvantages.
Here, we present protocols we referred upon designing our protocol.

\subsection{Time division multiple access}
Time-division multiple access (TDMA) is a protocol, which assigns ``time-slot'' exclusive to each terminal, along with a ``guard-time'' between each time slot to avoid overlapping of transmission, when two or more users have different propagation delays.
During the assigned time-slot, each terminal is allowed to do any kinds of operations.

However, since TDMA divides time, a synchronisation between terminals is required.
Also, each terminal in TDMA must be pre-configured which time-slot to use, making it useless in open environments.

\subsection{Ethernet for Control Automation Technology}
Ethernet for Control Automation Technology (EtherCAT) is a protocol where a network controller controls the whole network.
Data is interchanged by circulating frames transmitted by the controller in the network.
Only when a frame arrives, each terminal is allowed to read/write data.
After the node finished reading/writing, the node transmits the frame to the next node.
By performing this read/write/transmit procedure are on-the-fly, high-speed, and real-time communication is possible.

However, since each terminal in EtherCAT must be pre-configured, it is also useless in open environments.
Also, EtherCAT is a protocol for wired telecommunication, requiring a daisy-chain connection.

\section{Proposed protocol}
Our protocol is a connection-oriented full-duplex protocol, using the lightwave channel.
It involves several terminals, which communicate simultaneously with each other using the established session, and a central controller, to manage the network and sessions, as illustrated in \ref{fig:overview}.

\begin{figure}[tb]
  \centering
  \includegraphics[width=\linewidth]{overview-crop.pdf}
  \caption{\label{fig:overview}
    Illustration of proposed protocol's system.
  }
\end{figure}

The controller gives each terminal in the network two blocks, a unit of communication used in our protocol.
One is used to exchange data between the terminal to the controller (uplink).
The other is for the controller to the terminal (downlink).
Each terminal receives data from the given downlink block, and writes to its uplink block on the fly, building a single frame structure, as shown in \ref{fig:frame}, as a whole network.
This behaviour, based on EtherCAT, is to relish the decreased delay EtherCAT offers since each terminal does not have to process the entire frame to send its data.
However, unlike EtherCAT, where all terminals have to are wired in a daisy chain, our protocol is wireless.
\begin{figure}[tb]
  \centering
  \includegraphics[width=\linewidth]{frame-crop.pdf}
  \caption{\label{fig:frame}
    Illustration of frame used in the protocol.
  }
\end{figure}

Since we are dealing with open environments, each terminal must be able to join/leave the network and establish/demolish its sessions.
By sending requests to the controller (uplink request), using a contention-window-like request section, introduced for this specific purpose, we achieve an open environment behaviour.
All requests are defined in the request protocol, along with an identifier to reduce transmission, as shown in \ref{tab:request}.
The controller processes these requests and responds (downlink request) along with additional information.
We split the request section, for the downlink request not to collide with the uplink request since downlink requests transmit information to maintain a stable, predictable network.
On the other side, uplink requests are vulnerable to collisions; if many terminals make uplink requests at the same time, requests will collide and lead to unsuccessful uplink requests.
This problem is solved by each terminal retrying the same request after waiting a random amount of time.
\begin{table}[tb]
  \centering
  \small
  \caption{\label{tab:request}
    Table of request protocols.
  }
  \begin{tabularx}{\linewidth}{ccl}
    \hline
    Identifier & Name & Description\\
    \hline \hline
    R & Register & Join network\\
    U & Unregister & Leave network\\
    D & Connect & Connect to another terminal\\
    H & Disconnect & Break current connection\\
    \hline
  \end{tabularx}
\end{table}

\section{Implementation}
To satisfy the above-mentioned protocol, a System on a Chip (SoC) implementation, especially Zynq\textregistered -7000 SoC from Xilinx\textregistered, was chosen.

The controller and terminal is shown in \ref{fig:controller} and \ref{fig:terminal} respectivly.
Both the controller and terminal operates data by opening the data gate at the exact time.
Both operate with its oscillator; however, for terminals phase synchronisation is necessary.

\begin{figure}[tb]
  \centering
  \includegraphics[width=\linewidth]{NL-crop.pdf}
  \caption{\label{fig:controller}
    Illustration of controller.
  }
\end{figure}

\begin{figure}[tb]
  \centering
  \includegraphics[width=\linewidth]{UE-crop.pdf}
  \caption{\label{fig:terminal}
    Illustration of terminal.
  }
\end{figure}

The Processing System (ARM\textregistered -based processor) takes care of the received data, where for the controller and terminal are processed.
%% as \ref{fig:controller-proc} and \ref{fig:terminal-proc}, respectively.

\section{Evaluation}
Two elaluation was done to validate the protocol.
\begin{itemize}
\item \textbf{Evaluation 1}, validated the open enviroment performance of the protocol.
  Using a terminal (Terminal A) and controller, we verified that the terminal registered to the network.
  Upon un-registration, we also verified that the unused frames were released.
  \ref{fig:register} shows the registering process.
  \begin{figure}[tb]
    \centering
    \includegraphics[width=.75\linewidth]{register-crop.pdf}
    \caption{\label{fig:register}
      Diagram of the registration process.
      Upon successful registration, the controller successfully increases the number of devices in the network.
    }
  \end{figure}
\item \textbf{Evaluation 2}, validated the multiuser performance of the protocol.
  Using two terminals with an established session, we verified that two terminals could simultaneously send/receive datum.
    \begin{figure}[tb]
    \centering
    \includegraphics[width=\linewidth]{message-crop.pdf}
    \caption{\label{fig:register}
      Diagram of the registration process.
      Upon a successful registeration, the controller successfully increases the number of devices in the network.
    }
  \end{figure}    
\end{itemize}

\section{Conclusion}
This thesis presented a new protocol, based on TDMA and EtherCAT, aimed to enable multiuser telecommunication in an open environment with low symbol rate systems.
We introduced a central controller to enable multiuser specifications.
To maximise the bit efficiency, we removed the guard time and headers for each terminal.
To realise these specifications, we implemented the system to an SoC.
We validated the implementation and verified that the specifications meet the protocol specifications.
\end{document}
