\documentclass[twocolumn,9pt]{ltjsarticle}
\usepackage{graphicx}
\usepackage{amsmath,amssymb}
\usepackage{mathtools}
	\mathtoolsset{showonlyrefs=true}
\usepackage{./midterm-abstract}
% Lualatexの処理系の問題で、siunitxを使う場合は\usepackage{./midterm-abstract}の後にusepackageする
\usepackage[detect-all]{siunitx}
\usepackage{tabularx}
\usepackage{mwe}
\usepackage{cleveref}
\renewcommand{\ref}{\Cref}

\def\thesisnumber{2018-XX-YY}
\title{和文題目}
\def\englishtitle{English Title}
\author{8ZZZZZZZ 慶應太郎 (Taro Keio) Supervisor: 矢上諭吉 (Yukichi Yagami)}

\begin{document}
\maketitle
\section{Introduction}
Due to the dramatic increase in the number of connected and mobile communication devices, collecting and sending even more data than ever before has supported the growth of new technological areas, such as big data analysis and machine intelligence (a.k.a artificial intelligence).
Behind this spectacular growth, we are facing a significant challenge; data traffic expected to grow continuously, from \SI{122}{EB} per month in 2017 to a predicted \SI{396}{EB} in 2022 \cite{cisco}.
Also, traffic from wireless and mobile devices predicted to increase, from 52\% in 2017 to a predicted 71\% in 2022; satisfying greater spectral efficiency to enable even higher data rates and energy efficiency and low latency with the current wireless bands are critical, however impossible.

The International Telecommunication Union (ITU) defines a broad range of radio bands in its radio regulations,
from extremely-low frequency (ELF) band starting from 3--\SI{30}{Hz} to tremendously-high-frequency band at 300--\SI{3000}{GHz}.
Still, almost all current communication relies on the ultra-high frequency (UHF) at 300--\SI{3000}{MHz}, due to its convenient characteristics; penetrating foliage and buildings for indoor reception.
However, this UHF band is overloaded, many tweaks already introduced to use its bandwidth to the limit. 
Hence, they will not be able to handle the increase in traffic, as mentioned above.

Recently there has been a focused adoption of fifth-generation (5G) network, using millimetre waves or super-high to extremely-high frequencies; many countries are rushing for 5G as if it is the silver bullet to solve this problem.
However, since the vital challenges to adopting the millimetre waves, and due to many regulations with 5G, already a post-millimetre-wave communication utilising our oldest method for far distance communication, with the help of modern semiconductors are considered.

Lightwave communication originates from smoke signals and semaphores with flag or light (latter a.k.a. signal lamp) and developed into a ``Photophone'' by Bell in 1880 \cite{bell}.
Due to its extremely high line of sight and most importantly, the lack of technology to enable efficient and low power switching of lightwave signals to match that of radio wave, it did not attract much attention till recent years.
However, thanks to advances in modern semiconductors, and mass production of light-emitting diodes (LED) and photodiodes (PD), the field of lightwave communication has re-opened.
Currently playing the most crucial role in fibre-optic communication, lightwave is reaching for the wireless.

Modern wireless communication heavily relies on the modulation of LEDs \cite{nakagawa2004}.
Though the initial data rates were low, it has dramatically improved;
research by Bian et al. in 2017, using generalised space shift keying visible light communication achieved a data rate of \SI{800}{Mbps} with a bit error ratio of 0.02 \cite{Bian2017}.
Nevertheless, due to its high-speed modulation, lightwave communication relies on expensive and specialised modules, being the major roadblock for lightwave communication.

One cheapest and common lightwave communication module exists; the infrared module, widely used in remote controllers for consumer electronics, and previously used as an inter-device communication for feature phones.
Although, since infrared communication runs at a low modulation frequency of \SI{38}{kHz}, obtaining a compromisable data rate is a challenge, and since then, it has not drawn much attention.
Nonetheless, we argue that by carefully considering the protocol, or data frame structure of transmission, we can achieve a reasonable data rate, with multiuser characteristics.
Multiuser characteristics are essential when considering an open environment, where the number of communicating devices changes dynamically.

In this paper, we propose a tightly packed frame protocol, enabling multiuser access in an open environment.
Our system is implemented as a system on a chip (SoC), enabling us an appropriate balance of performance and flexibility.

The remainder of this paper is as follows;
Section \ref{sec:prp} presents our frame structure for our lightwave network, followed with implementation in Section \ref{sec:impl}.
Finally, we conclude this paper with Section \ref{sec:cncl}.

\section{Proposed protocol}
The protocol requires several terminals, which communicate simultaneously with each other in the network, and a central controller, to manage the network, as illustrated in \ref{fig:overview}.

\begin{figure}[tb]
  \centering
  \includegraphics[width=\linewidth]{overview-crop.pdf}
  \caption{\label{fig:overview}
    Illustration of proposed protocol's system.
  }
\end{figure}

Since we are dealing with open environments, each terminal joins/leaves the network, and connects/disconnects to another by sending requests to the controller.
The controller then processes requests and responds whether the request was accepted or not, along with additional information.
The protocol pre-define the requests as request protocols, as shown in \ref{tab:request}.

\begin{table}[tb]
  \centering
  \caption{\label{tab:request}
    Table of request protocols.
  }
  \begin{tabularx}{\linewidth}{ccl}
    \hline
    Identifier & Name & Description\\
    \hline \hline
    R & Register & Join network\\
    U & Unregister & Leave network\\
    D & Connect & Connect to another terminal\\
    H & Disconnect & Break current connection\\
    \hline
  \end{tabularx}
\end{table}

To achieve the specification, as mentioned earlier, we use an especial frame structure, as shown in \ref{fig:frame}.
The frame is TDMA like consisting of data, from multiple terminals and controller.
However, there are crucial differences.
\begin{itemize}
\item A footer-like request section is introduced.
  Terminals/Controller use this section to send/receive requests, defined in the request protocol.
\item Every block has no guard intervals; thus, data every block is entirely used as data.
\item The destination of each block is pre-defined by the network; thus, no extra headers/footers are needed in each block.
\end{itemize}

\begin{figure}[tb]
  \centering
  \includegraphics[width=\linewidth]{frame-crop.pdf}
  \caption{\label{fig:frame}
    Illustration of frame used in the protocol.
  }
\end{figure}

\subsection{Header}
The header consists of a single octet field, containing the number of connected terminals $n_{c}$ in the network.
Since each data block length, $l_d$ is fixed, and the number of data depends on the number of connected terminals, and request section is fixed length $l_r$, the frame length $L$ can be calculated by,
\begin{equation}
  \label{}
  L = n_c \cdot l_d \cdot 2 + l_r \cdot 2,
\end{equation}
which is used to detemine the end of the frame.

\subsection{Data}
The data consists of multiple four-octet blocks.
For each connected terminal in the network, the controller grants two data fields; one for the terminal to the controller (uplink) and the other for the controller to the terminal (downlink).
A four octet block is chosen since it can send one single-precision floating-point number, which the goal of this protocol is for motion controlling in open environments.

\subsection{Request}
The request consists of two double octet fields, each for uplink and downlink request.
This request field does every operation related to the network.
The operations are shown in \ref{tab:request}.

Only the controller performs downlink requests; however, many terminals make uplink requests, leading to collisions and unsuccessful uplink requests.

Each terminal solves this problem by performing a re-request after waiting a random amount of time (like ALOHA).
\section{Implementation}
To satisfy the above-mentioned protocol, a System on a Chip (SoC) implementation, especially Zynq\textregistered -7000 SoC from Xilinx\textregistered, was chosen.

The controller and terminal is shown in \ref{fig:controller} and \ref{fig:terminal} respectivly.
Both the controller and terminal operates data by opening the data gate at the exact time.
Both operate with its oscillator; however, for terminals phase synchronisation is necessary.

The Processing System (ARM\textregistered -based processor) takes care of the received data, where for the controller and terminal are processed as \ref{fig:controller-proc} and \ref{fig:terminal-proc}, respectively.

\section{Evaluation}
Two elaluation was done to validate the protocol.
\begin{itemize}
\item \textbf{Evaluation 1}, validated the open enviroment performance of the protocol.
  Using two terminals and controller, we verified that two terminals registered to the network, provided by the controller.
  Upon un-registration, we also verified that the unused frames were released.
\item \textbf{Evaluation 2}, validated the multiuser performance of the protocol.
  Using two terminals registered to the network, we verified that two terminals could simultaneously send/receive datum.
\end{itemize}

\section{Conclusion}
This thesis presented a new protocol, based on TDMA and EtherCAT, aimed to enable multiuser telecommunication in an open environment with low symbol rate systems.
We introduced a central controller to enable multiuser specifications.
To maximise the bit efficiency, we removed the guard time and headers for each terminal.
To realise these specifications, we implemented the system to an SoC.
We validated the implementation and verified that the specifications meet the protocol specifications.
\end{document}
